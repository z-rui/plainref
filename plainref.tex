\input macros

\centerline{\bf \TeX\ 参考卡}
\centerline{\sevenrm (适用于 Plain \TeX)}

\beginsection 希腊字母

\symalign{
alpha      &iota                   &varrho   \cr
beta       &kappa                  &sigma    \cr
gamma      &lambda                 &varsigma \cr
delta      &mu                     &tau      \cr
epsilon    &nu                     &upsilon  \cr
varepsilon &xi                     &phi      \cr
% o is a special case: it is not a control sequence.
zeta       &\specialsym{$o$}{\tt o}&varphi   \cr
eta        &pi                     &chi      \cr
theta      &varpi                  &psi      \cr
vartheta   &rho                    &omega    \cr
\noalign{\smallskip}
Gamma      &Xi                     &Phi      \cr
Delta      &Pi                     &Psi      \cr
Theta      &Sigma                  &Omega    \cr
Lambda     &Upsilon                          \cr
}

\beginsection Ord 类型符号

\symalign{
aleph   &prime     &forall         \cr
hbar    &emptyset  &exists         \cr
imath   &nabla     &neg \aka\lnot  \cr
jmath   &surd      &flat           \cr
ell     &top       &natural        \cr
wp      &bot       &sharp          \cr
Re      &|         &clubsuit       \cr
Im      &angle     &diamondsuit    \cr
partial &triangle  &heartsuit      \cr
infty   &backslash &spadesuit      \cr
}

\beginsection 大型运算符

\symalign{
sum     &bigcap   &bigodot   \cr
prod    &bigcup   &bigotimes \cr
coprod  &bigsqcup &bigoplus  \cr
int     &bigvee   &biguplus  \cr
oint    &bigwedge            \cr
}

\beginsection 二元运算符

\symalign{
pm       &cap             &vee   \aka\lor   \cr
mp       &cup             &wedge \aka\land  \cr
setminus &uplus           &oplus            \cr
cdot     &sqcap           &ominus           \cr
times    &sqcup           &otimes           \cr
ast      &triangleleft    &oslash           \cr
star     &triangleright   &odot             \cr
diamond  &wr              &dagger           \cr
circ     &bigcirc         &ddagger          \cr
bullet   &bigtriangleup   &amalg            \cr
div      &bigtriangledown                  \cr
}

\beginsection 页面布局

\csalign{
hsize=$\<尺寸>$&        设置页面宽度     \cr
vsize=$\<尺寸>$&        设置页面高度     \cr
displaywidth=$\<尺寸>$& 设置数学显示宽度 \cr
hoffset=$\<尺寸>$&      横向移动页面     \cr
voffset=$\<尺寸>$&      纵向移动页面     \cr
}

\beginsection 关系运算符

\symalign{
leq \aka\le &geq \aka\ge &equiv        \cr
prec        &succ        &sim          \cr
preceq      &succeq      &simeq        \cr
ll          &gg          &asymp        \cr
subset      &supset      &approx       \cr
subseteq    &supseteq    &cong         \cr
sqsubseteq  &sqsupseteq  &bowtie       \cr
in          &notin       &ni \aka\owns \cr
vdash       &dashv       &models       \cr
smile       &mid         &doteq        \cr
frown       &parallel    &perp         \cr
propto                                 \cr
\noalign{\medskip
\hbox{大部分关系运算符可以通过添加 \cs{not} 前缀得到相反的符号。}
\smallskip}
\specialsym{$\not\equiv$}{\cs{not}\cs{equiv}}&notin &ne \cr
}

\beginsection 箭头

\symalign{
leftarrow  \aka\gets &longleftarrow      \cr
Leftarrow            &Longleftarrow      \cr
rightarrow \aka\to   &longrightarrow     \cr
Rightarrow           &Rightarrow         \cr
leftrightarrow       &longleftrightarrow \cr
Leftrightarrow       &Longleftrightarrow \cr
mapsto               &longmapsto         \cr
hookleftarrow        &hookrightarrow     \cr
uparrow              &Uparrow            \cr
downarrow            &Downarrow          \cr
updownarrow          &Updownarrow        \cr
nearrow              &searrow            \cr
nwarrow              &swarrow            \cr
}\smallskip
\cs{buildrel} 宏将一个符号放置在另一个符号上方。
格式为 \cs{buildrel}\:$\<上标>$\:\cs{over}\:$\<关系>$。
\symalign{
\specialsym{\hfil$\buildrel\alpha\beta\over\longrightarrow$}%
{\cs{buildrel}\cs{alpha}\cs{beta}\cs{over}\cs{longrightarrow}}\cr
\specialsym{\hfil$f(x)\; {\buildrel\rm def\over=} \;x+1$}%
{\tt f(x)\cs; \char`\{\cs{buildrel}\cs{rm} def\cs{over}=\char`\} \cs;x+1}\cr
}

\beginsection 分界符

\def\ddelim#1{\specialsym{$#1\!#1$}{\tt\string#1\string\!\string#1}}
\symalign{
lbrack \aka[&lbrace \aka\{&langle         \cr
rbrack \aka[&rbrace \aka\{&rangle         \cr
vert   \aka|&lfloor       &lceil          \cr
Vert  \aka\|&rfloor       &rceil          \cr
\ddelim[&    \ddelim(&     \ddelim\langle \cr
\ddelim]&    \ddelim)&     \ddelim\rangle \cr
}\smallbreak
左右分界符加上前缀 \cs{left} 或 \cs{right} 后可以放大。
每个 \cs{left} 必须有一个对应的 \cs{right}, 其中一个可以是空的
(\cs{left.} 或 \cs{right.})。
使用下列命令指定一个特定的大小。

\centerline{\strut
\cs{bigl}, \cs{bigr} \hss \cs{Bigl}, \cs{Bigr} \hss \cs{biggl}, \cs{biggr}}

使用 \cs{bigm} 生成位于公式中间的大分界符, 或使用 \cs{big} 生成大的普通符号。

\beginsection 每次插入

\csalign{
everypar&     每当段落开始时插入             \cr
everymath&    每当文字中的数学内容开始时插入 \cr
everydisplay& 每当数学显示内容开始时插入     \cr
everycr&      每个 \cs{cr} 后插入            \cr
}

\beginsection 变音标记

\accalign{
帽子&      \hat a&         \cs ^\cr
可伸展帽子&\widehat{abc}&  无\cr
抑扬符&    \check a&       \cs v\cr
波浪&      \tilde a&       \cs ~\cr
可伸展波浪&\widetilde{abc}&无\cr
锐音符&    \acute a       &\cs '\cr
抑音符&    \grave a       &\cs `\cr
点&        \dot a         &\cs .\cr
两点&      \ddot a        &\cs "\cr
短音符&    \breve a       &\cs u\cr
横线&      \bar a         &\cs =\cr
向量&      \vec a         &无\cr
}\smallskip
\cs{skew}$\<数字>$ 命令移动变音符到合适的位置, $\<数字>$越大,
越向右移。 比较:
$$
\hbox{\tt\cs{hat}\braced{\cs{hat} A} 生成 $\hat{\hat A}$},
\qquad
\hbox{\tt\cs{skew}6\cs{hat}\braced{\cs{hat} A} 生成 $\skew6\hat{\hat A}$}.
$$

\beginsection 初等数学控制序列

\halign to\hsize{%
#\hfil\tabskip\centering&\hfil$\displaystyle{#}$\hfil&\tt#\hfil\tabskip=0pt\cr
上划线&\overline{x+y}&\cs{overline}\braced{x+y}\cr
下划线&\underline{x+y}&\cs{underline}\braced{x+y}\cr
平方根&\sqrt{x+2}&\cs{sqrt}\braced{x+2}\cr
高次方根&\root n\of{x+2}&\cs{root}\ n\cs{of}\braced{x+2}\cr
分式&n+1\over3&\braced{n+1\cs{over}3}\cr
分式 (无分数线)&n+1\atop 3&\braced{n+1\cs{atop} 3}\cr
二项式系数&n+1\choose 3&\braced{n+1\cs{choose} 3}\cr
大括号分式&n+1\brace 3&\braced{n+1\cs{brace} 3}\cr
方括号分式&n+1\brack 3&\braced{n+1\cs{brack} 3}\cr
}\smallskip

以下指令用于指定公式排版风格。

\centerline{\cs{displaystyle} \cs{textstyle}
\cs{scriptstyle} \cs{scriptscriptstyle}}

\beginsection 正体字函数名

\halign to\hsize{&\cs{#}\hfil&\amp#\tabskip\centering\cr
arccos&cos& csc&exp&ker&   limsup&min&sinh\cr
arcsin&cosh&deg&gcd&lg&    ln&    Pr& sup\cr
arctan&cot& det&hom&lim&   log&   sec&tan\cr
arg&   coth&dim&inf&liminf&max&   sin&tanh\cr
}\smallskip
\halign to\hsize{\tt#\hfil\tabskip\centering&$#$\hfil&#\hfil\tabskip=0pt\cr
a \cs{pmod}\braced{m}&a\pmod m&带括号 mod\cr
a \cs{bmod} m&a\bmod m&不带括号 mod\cr
}\medskip
以下例子使用 \cs{mathop} 创建函数名。
\halign to\hsize{%
$\displaystyle{#}$\hfil\tabskip\centering&\tt#\hfil&\tt#\hfil\tabskip=0pt\cr
\omit 例子\hfil&\omit 命令\hfil&\omit Plain \TeX\ 定义\hfil\cr
\lim_{x\to2}&\cs{lim}\string_\braced{x\cs{to2}}&
\cs{def}\cs{lim}\braced{\cs{mathop}\braced{\cs{rm} lim}}\cr
\log_2&\cs{log}\string_2&
\cs{def}\cs{log}\braced{\cs{mathop}\braced{\cs{rm} log}\cs{nolimits}}\cr
}

\beginsection 脚注、 插入和下划线

\csalign{
footnote$\<标记>$\braced{$\<文本>$}&     脚注\cr
topinsert$\<竖模式内容>$\cs{endinsert}&  在页首插入\cr
pageinsert$\<竖模式内容>$\cs{endinsert}& 在整页插入\cr
midinsert$\<竖模式内容>$\cs{endinsert}&  在页中插入\cr
underbar\braced{$\<文本>$}&              下划线文本\cr
}\bigskip

\begingroup \baselineskip=10pt \sevenrm
\centerline{$\scriptstyle\copyright$ 2007 J.H.~Silverman, 授权在背面。 v1.5}
发送评论和更正至 J.H.~Silverman, Math.~Dept., Brown Univ.,
Providence, RI 02912 USA.
$\scriptstyle\langle\hbox{jhs@math.brown.edu}\rangle$
\par\endgroup

\eject

\beginsection 实用参数和转换

\csalign{
day,\cs{month},\cs{year}&        当前日、 月、 年\cr
jobname&                         当前任务名\cr
romannumeral$\<数学>$&           转换为小写罗马数字\cr
uppercase\braced{$\<记号列表>$}& 转换为大写\cr
lowercase\braced{$\<记号列表>$}& 转换为小写\cr
}

\beginsection 填充、 导引和省略号

\symalign{
\omit\span\omit 文本或数学:\hfil&dots \cr
\omit 数学:\quad$\ldots$\hfil   &\cs{ldots}&cdots &vdots &ddots \cr
}\smallskip
以下命令以相应内容填充空间。

\centerline{\cs{hrulefill} \cs{rightarrowfill} \cs{leftarrowfill} \cs{dotfill}}
\smallskip

构造导引的一般格式为
\csalign{
leaders$\<盒或线>$\cs{hskip}$\<胶>$& 重复盒或线\cr
leaders$\<盒或线>$\cs{hfill}&        用盒或线填充\cr
}

\beginsection \TeX\ 字体和缩放

\halign to\hsize{&\cs{#}\quad\hfil&#\hfil\tabskip\centering\cr
rm&\rm Roman&bf&\bf Bold&tt&\tt Typewriter\cr
sl&\sl Slant&it&\it Italic&/&“斜体修正”\cr
}\smallskip
\csalign{
magnification=$\<数字>$&              缩放文档 $n/1000$ 倍\cr
magstep$\<数字>$&                     缩放因数 $1.2^n\times 1000$\cr
magstephalf&                          缩放因数 $\sqrt{1.2}$\cr
font\cs{FN}=$\<字体名>$&              载入字体, 命名为 \cs{FN}\cr
font\cs{FN}=$\<字体名>$ at $\<尺寸>$&
                            \hphantom{载入字体, }缩放到尺寸\cr
font\cs{FN}=$\<字体名>$ scaled $\<数字>$&
                            \hphantom{载入字体, }缩放 $n/1000$ 倍\cr
\omit\tt true$\<尺寸>$\hfil&          不缩放的尺寸\cr
char`\cs{c}&                          打印字符或符号 $c$\cr
}

\beginsection 对齐显示

\csalign{
settabs$\<数字>$\cs{columns}&                   设置等宽制表位\cr
settabs\cs{+}$\<样例行>$&                       根据样例行设置制表位\cr
+$\<文本$_1$>$\&$\<文本$_2$>$\&$\cdots$\cs{cr}& 要排版的制表文本\cr
halign&                                         横向对齐\cr
halign to$\<尺寸>$&                             横向对齐\cr
openup$\<尺寸>$&                                增加行距\cr
noalign\braced{$\<竖模式内容>$}&                在 \cs{cr} 后插入内容\cr
tabskip=$\<胶>$&                                设置制表位处的胶\cr
omit&                                           省略一列的模板\cr
span&                                           合并两列\cr
multispan$\<数字>$&                             合并若干列\cr
hidewidth&                                      忽略一个单元格的宽度\cr
crcr&                                           插入 \cs{cr} 如果不存在\cr
}

\beginsection 盒

\csalign{
hbox to$\<尺寸>$&    给定尺寸的横向盒\cr
vbox to$\<尺寸>$&    纵向盒, 底端对齐\cr
vtop to$\<尺寸>$&    纵向盒, 顶端对齐\cr
vcenter to$\<尺寸>$& 纵向盒, 中间对齐 (仅限数学模式)\cr
rlap&                右侧覆盖内容\cr
llap&                左侧覆盖内容\cr
}

\beginsection 过满的盒

\csalign{
hfuzz&         横向盒中允许超出的量\cr
vfuzz&         纵向盒中允许超出的量\cr
overfullrule&  过满的盒的标记宽度。 要完全消除,\cr % 接下一行
\omit&         \qquad 设置 \cs{overfullrule=0pt}。\cr
}

\beginsection 缩进和项目列表

\csalign{
indent&                     缩进\cr
noindent&                   不缩进\cr
parindent=$\<尺寸>$&        设置首行缩进\cr
displayindent=$\<尺寸>$&    设置数学显示缩进\cr
leftskip=$\<尺寸>$&         左侧空白\cr
rightskip=$\<尺寸>$&        右侧空白\cr
narrower&                   使段落更窄\cr
item\braced{$\<标签>$}&     一级缩进项目列表\cr
itemitem\braced{$\<标签>$}& 二级缩进项目列表\cr
hangindent=$\<尺寸>$&       段落悬挂缩进\cr
hangafter=$\<数字>$&        在第 $n$ 行后开始悬挂缩进。\cr % 接下一行
\omit&                      \qquad 如果 $n<0$, 缩进前 $|n|$ 行。\cr
parshape=$\<数字>$&         一般段落形状命令\cr
}

\beginsection 页眉、 页脚和页码

\csalign{
nopagenumbers& 关闭页码\cr
pageno&        当前页码。 设置 \cs{pageno=$\<负数>$}以得到罗马数字,\cr
folio&         当前页码, 如果 $<0$ 则得到罗马数字\cr
footline&      放在页脚的内容\cr
headline&      放在页眉的内容。 要留出空间, 设置\cr
\omit&         \quad {\tt\cs{voffset}=2\cs{baselineskip}}, 并用\cr
\omit&         \quad {\tt\cs{advance}\cs{vsize} by-\cs{voffset}} 创造空间\cr
}

\beginsection 宏定义

\csalign{
def\cs{cs}\braced{$\<替代文本>$}&定义宏 \cs{cs}\cr
def\cs{cs}\#1$\cdots$\#$n$\braced{$\<替代文本>$}&带参数的宏\cr
let\cs{cs}=$\<记号>$&令 \cs{cs} 为记号的当前含义\cr
}\smallskip

\leftline{\bf 高级宏定义命令}
\csalign{
long\cs{def}&                      参数可以包含 \cs{par} 的宏\cr
outer\cs{def}&                     不允许出现在定义中的宏\cr
global\cs{def}{\rm~或~}\cs{gdef}&  定义跨越分组的宏\cr
edef&                              定义宏时展开\cr
xdef{\rm~或~}\cs{global}\cs{edef}& \cs{edef}的全局版本\cr
noexpand$\<记号>$&                 不展开记号\cr
expandafter$\<记号>$&              先展开记号后的项\cr
futurelet\cs{cs}$\<记号$_1$>\<记号$_2$>$\hidewidth&
  \hskip1.5em 等价于 \cs{let}\cs{cs=}$\<记号$_2$>\<记号$_1$>\<记号$_2$>$\cr
csname$\ldots$\cs{endcsname}&      创建控制序列名称\cr
string\cs{cs}&                     列出名称中的字符, {\tt\string\ c s}\cr
number$\<数字>$&                   数字中的字符列表\cr
the$\<内部量>$&                    给出量的值的记号序列\cr
}

\beginsection 条件指令

条件指令的一般格式为

\centerline{\hbox{\cs{if}$\<条件>\<真时文本>$\cs{else}$\<假时文本>$\cs{fi}}}

\csalign{
ifnum$\<数字$_1$>\<关系>\<数字$_2$>$& 比较两整数\cr
ifdim$\<尺寸$_1$>\<关系>\<尺寸$_2$>$& 比较两尺寸\cr
ifodd$\<数字>$&                       测试奇数\cr
ifmmode&                              测试数学模式\cr
if$\<记号$_1$>\<记号$_2$>$&           测试字符代码匹配\cr
ifx$\<记号$_1$>\<记号$_2$>$&          测试记号匹配\cr
ifdim$\<尺寸$_1$>\<尺寸$_2$>$&        测试尺寸匹配\cr
ifeof$\<数字>$&                       测试文件结束\cr
iftrue{\rm, }\cs{iffalse}&            始终真、 始终假\cr
% 第一列分两行
ifcase$\<数字>\<文本$_0$>$\cs{or}$\<文本$_1$>$\cs{or}$\cdots\hidewidth$\cr
\omit\quad \cs{or}$\<文本$_n$>$\cs{else}$\<文本>$\cs{fi}&
                                      根据$\<数字>$选择文本\cr
loop $\alpha$ \cs{if}$\ldots\beta$ \cs{repeat}&
                                      循环 $\alpha\beta\alpha\cdots\alpha$
				      直到 \cs{if} 为假\cr
newif\cs{ifblob}&                     创建名为 \cs{ifblob} 的新条件指令\cr
blobtrue{\rm, }\cs{blobfalse}&        设置条件 \cs{ifblob} 为真、 假\cr
}

\beginsection 尺寸、 间距和胶

尺寸用$\<数字>\<单位>$指定。

胶用$\tt\<尺寸>\;plus\<尺寸>\;minus\<尺寸>$指定。

\halign to\hsize{&#\hfil\tabskip=1em&
\tt#\hfil\tabskip=0pt&\amp#\tabskip\centering&\strut\vrule#\cr
点&  pt&&派卡&pc&&英寸&    in&&厘米&cm\cr
m 宽&em&&x 高&ex&&数学单位&mu&&毫米&mm\cr
\multispan2 1 pc = 12 pt&&
\multispan2 1 in = 72.27 pt&&
\multispan2 2.54 cm = 1 in&&
\multispan2 18 mu = 1 em\cr
}\smallskip

\begingroup
\def\item#1{\par\leavevmode\hbox to.5\hsize{#1\hss}\ignorespaces}
\def\itemitem#1{\item{\qquad#1}}
\def\;{\hskip 1em plus .5em minus .5em\ignorespaces}
\item{横向空格:} \cs{quad} (跳过 1em)\; \cs{quad}
\item{横向空格 (文本):} \cs{thinspace}\; \cs{enspace}\;
\cs{enskip}\; \cs{hskip$\<胶>$}\; \cs{hfil}\; \cs{hfill}\; \cs{hfilneg}
\item{横向空格 (数学):} 窄空格 \cs{,}\; 中等空格 \cs{>}\;
宽空格 \cs{;}\; 负窄空格 \cs{!}\; \cs{mskip$\<mu 胶>$}
\smallskip
\item{纵向空格:} \cs{vskip$\<胶>$}\; \cs{vfil}\; \cs{vfill}
\halign{\qquad\cs{#}\hfil\tabskip=1em&#\hfil\tabskip=0pt\cr
strut&高和深相当于 ``('', 宽为零的盒\cr
phantom\braced{$\<文本>$}&具有$\<文本>$尺寸的不可见盒\cr
vphantom\braced{$\<文本>$}&高和深相当于$\<文本>$, 宽为零的盒\cr
hphantom\braced{$\<文本>$}&宽相当于$\<文本>$, 高和深为零的盒\cr
smash\braced{$\<文本>$}&排版$\<文本>$, 设置高和宽为零\cr
}
\halign{%
\qquad\cs{#}$\<尺寸>$\cs{hbox}$\<文本>$\hfil\tabskip=1em&
向#移动盒\hfil\tabskip=0pt\cr
    raise&上\cr
    lower&下\cr
 moveleft&左\cr
moveright&右\cr
}
\item{行间空格:} \cs{smallskip}\; \cs{medskip}\; \cs{bigskip}
\itemitem{鼓励分页}\hidewidth \cs{smallbreak}\; \cs{medbreak}\; \cs{bigbreak}
\itemitem{空间不足时分页} \cs{filbreak}
\item{设置行间距:} \cs{baselineskip = $\<胶>$}
\itemitem{单倍行距} \cs{baselineskip = 12pt}
\itemitem{1.5 倍行距} \cs{baselineskip = 18pt}
\itemitem{双倍行距} \cs{baselineskip = 24pt}
\item{增加行间距} \cs{openup$\<尺寸>$}
\itemitem{使用 \cs{jot}} {\tt 1\cs{jot} = 3pt}
\item{允许行边界不对齐} \cs{raggedright}
\item{允许页边界不对齐} \cs{raggedbottom}
\par\endgroup

\beginsection 括弧和矩阵

\csalign{
matrix&       矩阵\cr
pmatrix&      带括号矩阵\cr
bordermatrix& 带上方和左侧标签的矩阵\cr
overbrace&    上花括号, 可以有上标\cr
underbrace&   下花括号, 可以有下标\cr
}\smallskip

对于文本中的小矩阵, 使用以下构造:
{\tabskip=2em\halign{\tt#\hfil&$#$\hfil\cr
\braced{a\cs{,}b \cs{choose} c\cs{,}d}&{a\,b\choose c\,d}\cr
\cs{left}( \braced{a\cs{atop} c} \braced{b\cs{atop} d} \cs{right})&
\left({a\atop c}{b\atop d}\right)\cr
}}

\beginsection 显示公式

\csalign{
eqno&         右侧方程编号\cr
leqno&        左侧方程编号\cr
eqalign&      显示若干对齐方程\cr
eqalignno&    显示右侧编号对齐方程\cr
leqalignno&   显示左侧编号对齐方程\cr
displaylines& 显示若干居中方程\cr
cases&        分类定义\cr
noalign&      在任意 \cs{cr} 后,
              用 {\tt\cs{noalign}\braced{\cs{vskip}$\<胶>$}} \cr % 接下一行
\omit&        \quad 在行间插入空格\cr
openup&       在显示的所有行间添加空格\cr
}\bigskip

\begingroup \baselineskip=10pt \sevenrm
\centerline{R.Z.~于 2018 年, 译自:}
\centerline{版权所有 $\scriptstyle\copyright$ 2007 J.H.~Silverman,
2007~年 1~月 v1.5}
\centerline{Math. Dept., Brown Univ. Providence, RI 02912 USA}
\centerline{\TeX 是美国数学学会的商标}
\centerline{允许制作和分发此卡片的副本,
条件是版权提示和此授权在所有副本上均以保留。}
\par\endgroup

\eject\end
